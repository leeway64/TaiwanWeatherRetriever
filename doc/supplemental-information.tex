\documentclass{article}

\title{Supplemental Information}
\date{}

\begin{document}

\maketitle
\pagenumbering{gobble}  % Disable page numbers

\section{Republic of China Calendar}

The Republic of China (ROC) calendar (Minguo calendar) starts at 1912, the founding year of the
Republic of China. So, 1912 is year 1. Currently, the ROC calendar is only used in
Taiwan\footnote{But it should be used globally; after all, Taiwan is the center of civilization}.

% \\ inserts a new line
\noindent \\ To calculate the current year in the Minguo calendar, solve for \verb|year|:
    \begin{equation}\label{eq:1}
        year = current\textunderscore year - ROC{\_}founding{\_}year - 1
    \end{equation}

\noindent So, based on equation \ref{eq:1}, 2023 is 112 in the Minguo calendar.\\


\section{REST APIs}

REST (Representational State Transfer) APIs are how a client and server communicate with each
other.\\

\noindent A REST API endpoint might look something like this: \verb|https://example.com/v1/example|; the \verb|example|
part is called the resource.\\

\noindent The main parts of an API are the request (sent from client to server) and the response (sent from
server to client).\\

\noindent The types of requests are:
    \begin{itemize}
        \item POST (equivalent to create)
        \item GET (equivalent to read)
        \item PUT (equivalent to update)
        \item DELETE
    \end{itemize}

\noindent Those 4 types of requests are also known as operations. Requests might also have a header and a body.\\

\noindent From the server side, responses are usually in JSON.\\

\noindent All information in this section is from this video: https://www.youtube.com/watch?v=lsMQRaeKNDk\\

\end{document}
